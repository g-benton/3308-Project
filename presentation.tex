\documentclass[10pt]{beamer}

\usetheme[progressbar=frametitle]{metropolis}
\usepackage{appendixnumberbeamer}

\usepackage{booktabs}
\usepackage[scale=2]{ccicons}

\usepackage{pgfplots}
\usepgfplotslibrary{dateplot}

\usepackage{xspace}
\newcommand{\themename}{\textbf{\textsc{metropolis}}\xspace}

\title{Graze}
\subtitle{Making food easier.}
% \date{\today}
\date{}
\author{Michael Barlow, Greg Benton, Rasheeq Jahan, Sofia Mehrotra, Erin Ruby}
\institute{University of Colorado}
% \titlegraphic{\hfill\includegraphics[height=1.5cm]{logo.pdf}}

\begin{document}

\maketitle

\begin{frame}{Table of contents}
  \setbeamertemplate{section in toc}[sections numbered]
  \tableofcontents[hideallsubsections]
\end{frame}

\section{Introduction}

\begin{frame}[fragile]{Ideology and Goal}

    Often the hardest part of meal time is asking "what should I make?", to eliminate this familiar culinary woe we wanted,
    \begin{itemize}
    	\item A clean, simple, efficient user interface

        \item To take in ingredients or main ideas

        \item To output a straightforward, one by one, list of possible recipes, eliminating the need to decide from a massive list of options
    \end{itemize}



\end{frame}
\begin{frame}[fragile]{Requirements}

	We made a list of requirements to make sure we met our goals of user-focused simplicity, including,
    \begin{itemize}
    	\item No user accounts or sign up necessary
        \item Easy to restart or refresh user progress
        \item Output user results in a simple and digestible format (i.e. one at a time)
    \end{itemize}

\end{frame}

\begin{frame}[fragile]{Methodology}
	We used the Agile methodology, slightly adapted to fit into the time constraints of the course.

    This meant that (starting at around the third week) we had a scrum nearly every day that class met.

    Sprints were formed on a shortened cycle, with a typical sprint lasting between 3-7 days depending on the scope of our goals
\end{frame}

\begin{frame}{Sprints}
	WHAT WERE OUR SPRINTS??
\end{frame}


\begin{frame}{Tools}
	We used the following resources to build our website,
    \begin{itemize}
    	\item \alert{Django} - Our main development platform and service
        \item \alert{MySQL} - Our database software
        \item \alert{Docker} - Our containerization service to allow everyone to develop with the same services
        \item \alert{Heroku} - Our platform for final web hosting
    \end{itemize}
\end{frame}

\section{Challenges}

\begin{frame}{Challenges: Initial Setup}
	One major challenge we faced was getting everyone on the same page with development. We are working on three different operating systems each with its own requirements. Docker was a major help with this, but we encountered an unforeseen delay in setting everyone up with a working development environment.
\end{frame}

\begin{frame}{Challenges: Data}
Getting meaningful data.
\begin{itemize}
	\item Could enter a small amount of data by hand, but will be limited and time consuming
    \item Found \textit{Yummly}, a large API that allowed us to build an initial database using python scripts and allows the database to grow as the website is used
\end{itemize}
\end{frame}

\begin{frame}{Challenges: }

\end{frame}

\section{Features and Functionality}

\begin{frame}{Feature: Ingredient Input}
The main function of the website is the user's ability to enter whatever ingredients they have on hand.

The home page of the website is where the user accesses this feature, bypassing any need for logging in or registering, keeping in line with our ease-of-use beliefs.
\end{frame}

\begin{frame}{Feature: Recipe Retrieval}
NOT SURE HOW THIS WILL LOOK, NEED TO WAIT
\end{frame}


\begin{frame}{Conclusion}
 GITHUB LINK, LESSONS LEARNED (friendship), RESULTS
\end{frame}

{\setbeamercolor{palette primary}{fg=black, bg=yellow}
\begin{frame}[standout]
  Questions?\\
  \tiny (Probably not)
\end{frame}
}




\end{document}
